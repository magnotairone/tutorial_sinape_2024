% Options for packages loaded elsewhere
\PassOptionsToPackage{unicode}{hyperref}
\PassOptionsToPackage{hyphens}{url}
\PassOptionsToPackage{dvipsnames,svgnames,x11names}{xcolor}
%
\documentclass[
  letterpaper,
  DIV=11,
  numbers=noendperiod]{scrartcl}

\usepackage{amsmath,amssymb}
\usepackage{iftex}
\ifPDFTeX
  \usepackage[T1]{fontenc}
  \usepackage[utf8]{inputenc}
  \usepackage{textcomp} % provide euro and other symbols
\else % if luatex or xetex
  \usepackage{unicode-math}
  \defaultfontfeatures{Scale=MatchLowercase}
  \defaultfontfeatures[\rmfamily]{Ligatures=TeX,Scale=1}
\fi
\usepackage{lmodern}
\ifPDFTeX\else  
    % xetex/luatex font selection
\fi
% Use upquote if available, for straight quotes in verbatim environments
\IfFileExists{upquote.sty}{\usepackage{upquote}}{}
\IfFileExists{microtype.sty}{% use microtype if available
  \usepackage[]{microtype}
  \UseMicrotypeSet[protrusion]{basicmath} % disable protrusion for tt fonts
}{}
\makeatletter
\@ifundefined{KOMAClassName}{% if non-KOMA class
  \IfFileExists{parskip.sty}{%
    \usepackage{parskip}
  }{% else
    \setlength{\parindent}{0pt}
    \setlength{\parskip}{6pt plus 2pt minus 1pt}}
}{% if KOMA class
  \KOMAoptions{parskip=half}}
\makeatother
\usepackage{xcolor}
\setlength{\emergencystretch}{3em} % prevent overfull lines
\setcounter{secnumdepth}{-\maxdimen} % remove section numbering
% Make \paragraph and \subparagraph free-standing
\ifx\paragraph\undefined\else
  \let\oldparagraph\paragraph
  \renewcommand{\paragraph}[1]{\oldparagraph{#1}\mbox{}}
\fi
\ifx\subparagraph\undefined\else
  \let\oldsubparagraph\subparagraph
  \renewcommand{\subparagraph}[1]{\oldsubparagraph{#1}\mbox{}}
\fi

\usepackage{color}
\usepackage{fancyvrb}
\newcommand{\VerbBar}{|}
\newcommand{\VERB}{\Verb[commandchars=\\\{\}]}
\DefineVerbatimEnvironment{Highlighting}{Verbatim}{commandchars=\\\{\}}
% Add ',fontsize=\small' for more characters per line
\usepackage{framed}
\definecolor{shadecolor}{RGB}{241,243,245}
\newenvironment{Shaded}{\begin{snugshade}}{\end{snugshade}}
\newcommand{\AlertTok}[1]{\textcolor[rgb]{0.68,0.00,0.00}{#1}}
\newcommand{\AnnotationTok}[1]{\textcolor[rgb]{0.37,0.37,0.37}{#1}}
\newcommand{\AttributeTok}[1]{\textcolor[rgb]{0.40,0.45,0.13}{#1}}
\newcommand{\BaseNTok}[1]{\textcolor[rgb]{0.68,0.00,0.00}{#1}}
\newcommand{\BuiltInTok}[1]{\textcolor[rgb]{0.00,0.23,0.31}{#1}}
\newcommand{\CharTok}[1]{\textcolor[rgb]{0.13,0.47,0.30}{#1}}
\newcommand{\CommentTok}[1]{\textcolor[rgb]{0.37,0.37,0.37}{#1}}
\newcommand{\CommentVarTok}[1]{\textcolor[rgb]{0.37,0.37,0.37}{\textit{#1}}}
\newcommand{\ConstantTok}[1]{\textcolor[rgb]{0.56,0.35,0.01}{#1}}
\newcommand{\ControlFlowTok}[1]{\textcolor[rgb]{0.00,0.23,0.31}{#1}}
\newcommand{\DataTypeTok}[1]{\textcolor[rgb]{0.68,0.00,0.00}{#1}}
\newcommand{\DecValTok}[1]{\textcolor[rgb]{0.68,0.00,0.00}{#1}}
\newcommand{\DocumentationTok}[1]{\textcolor[rgb]{0.37,0.37,0.37}{\textit{#1}}}
\newcommand{\ErrorTok}[1]{\textcolor[rgb]{0.68,0.00,0.00}{#1}}
\newcommand{\ExtensionTok}[1]{\textcolor[rgb]{0.00,0.23,0.31}{#1}}
\newcommand{\FloatTok}[1]{\textcolor[rgb]{0.68,0.00,0.00}{#1}}
\newcommand{\FunctionTok}[1]{\textcolor[rgb]{0.28,0.35,0.67}{#1}}
\newcommand{\ImportTok}[1]{\textcolor[rgb]{0.00,0.46,0.62}{#1}}
\newcommand{\InformationTok}[1]{\textcolor[rgb]{0.37,0.37,0.37}{#1}}
\newcommand{\KeywordTok}[1]{\textcolor[rgb]{0.00,0.23,0.31}{#1}}
\newcommand{\NormalTok}[1]{\textcolor[rgb]{0.00,0.23,0.31}{#1}}
\newcommand{\OperatorTok}[1]{\textcolor[rgb]{0.37,0.37,0.37}{#1}}
\newcommand{\OtherTok}[1]{\textcolor[rgb]{0.00,0.23,0.31}{#1}}
\newcommand{\PreprocessorTok}[1]{\textcolor[rgb]{0.68,0.00,0.00}{#1}}
\newcommand{\RegionMarkerTok}[1]{\textcolor[rgb]{0.00,0.23,0.31}{#1}}
\newcommand{\SpecialCharTok}[1]{\textcolor[rgb]{0.37,0.37,0.37}{#1}}
\newcommand{\SpecialStringTok}[1]{\textcolor[rgb]{0.13,0.47,0.30}{#1}}
\newcommand{\StringTok}[1]{\textcolor[rgb]{0.13,0.47,0.30}{#1}}
\newcommand{\VariableTok}[1]{\textcolor[rgb]{0.07,0.07,0.07}{#1}}
\newcommand{\VerbatimStringTok}[1]{\textcolor[rgb]{0.13,0.47,0.30}{#1}}
\newcommand{\WarningTok}[1]{\textcolor[rgb]{0.37,0.37,0.37}{\textit{#1}}}

\providecommand{\tightlist}{%
  \setlength{\itemsep}{0pt}\setlength{\parskip}{0pt}}\usepackage{longtable,booktabs,array}
\usepackage{calc} % for calculating minipage widths
% Correct order of tables after \paragraph or \subparagraph
\usepackage{etoolbox}
\makeatletter
\patchcmd\longtable{\par}{\if@noskipsec\mbox{}\fi\par}{}{}
\makeatother
% Allow footnotes in longtable head/foot
\IfFileExists{footnotehyper.sty}{\usepackage{footnotehyper}}{\usepackage{footnote}}
\makesavenoteenv{longtable}
\usepackage{graphicx}
\makeatletter
\def\maxwidth{\ifdim\Gin@nat@width>\linewidth\linewidth\else\Gin@nat@width\fi}
\def\maxheight{\ifdim\Gin@nat@height>\textheight\textheight\else\Gin@nat@height\fi}
\makeatother
% Scale images if necessary, so that they will not overflow the page
% margins by default, and it is still possible to overwrite the defaults
% using explicit options in \includegraphics[width, height, ...]{}
\setkeys{Gin}{width=\maxwidth,height=\maxheight,keepaspectratio}
% Set default figure placement to htbp
\makeatletter
\def\fps@figure{htbp}
\makeatother

\KOMAoption{captions}{tableheading}
\makeatletter
\makeatother
\makeatletter
\makeatother
\makeatletter
\@ifpackageloaded{caption}{}{\usepackage{caption}}
\AtBeginDocument{%
\ifdefined\contentsname
  \renewcommand*\contentsname{Índice}
\else
  \newcommand\contentsname{Índice}
\fi
\ifdefined\listfigurename
  \renewcommand*\listfigurename{Lista de Figuras}
\else
  \newcommand\listfigurename{Lista de Figuras}
\fi
\ifdefined\listtablename
  \renewcommand*\listtablename{Lista de Tabelas}
\else
  \newcommand\listtablename{Lista de Tabelas}
\fi
\ifdefined\figurename
  \renewcommand*\figurename{Figura}
\else
  \newcommand\figurename{Figura}
\fi
\ifdefined\tablename
  \renewcommand*\tablename{Tabela}
\else
  \newcommand\tablename{Tabela}
\fi
}
\@ifpackageloaded{float}{}{\usepackage{float}}
\floatstyle{ruled}
\@ifundefined{c@chapter}{\newfloat{codelisting}{h}{lop}}{\newfloat{codelisting}{h}{lop}[chapter]}
\floatname{codelisting}{Listagem}
\newcommand*\listoflistings{\listof{codelisting}{Lista de Listagens}}
\makeatother
\makeatletter
\@ifpackageloaded{caption}{}{\usepackage{caption}}
\@ifpackageloaded{subcaption}{}{\usepackage{subcaption}}
\makeatother
\makeatletter
\@ifpackageloaded{tcolorbox}{}{\usepackage[skins,breakable]{tcolorbox}}
\makeatother
\makeatletter
\@ifundefined{shadecolor}{\definecolor{shadecolor}{rgb}{.97, .97, .97}}
\makeatother
\makeatletter
\makeatother
\makeatletter
\makeatother
\ifLuaTeX
\usepackage[bidi=basic]{babel}
\else
\usepackage[bidi=default]{babel}
\fi
\babelprovide[main,import]{portuguese}
% get rid of language-specific shorthands (see #6817):
\let\LanguageShortHands\languageshorthands
\def\languageshorthands#1{}
\ifLuaTeX
  \usepackage{selnolig}  % disable illegal ligatures
\fi
\IfFileExists{bookmark.sty}{\usepackage{bookmark}}{\usepackage{hyperref}}
\IfFileExists{xurl.sty}{\usepackage{xurl}}{} % add URL line breaks if available
\urlstyle{same} % disable monospaced font for URLs
\hypersetup{
  pdftitle={Da Teoria à Prática: Modelos de IA Generativa com R e Python},
  pdfauthor={Magno T. F. Severino},
  pdflang={pt},
  colorlinks=true,
  linkcolor={blue},
  filecolor={Maroon},
  citecolor={Blue},
  urlcolor={Blue},
  pdfcreator={LaTeX via pandoc}}

\title{Da Teoria à Prática: Modelos de IA Generativa com R e Python}
\usepackage{etoolbox}
\makeatletter
\providecommand{\subtitle}[1]{% add subtitle to \maketitle
  \apptocmd{\@title}{\par {\large #1 \par}}{}{}
}
\makeatother
\subtitle{Consultando e Extraindo Informações de PDFs com IA}
\author{Magno T. F. Severino}
\date{2024-08-08}

\begin{document}
\maketitle
\ifdefined\Shaded\renewenvironment{Shaded}{\begin{tcolorbox}[borderline west={3pt}{0pt}{shadecolor}, interior hidden, sharp corners, frame hidden, breakable, enhanced, boxrule=0pt]}{\end{tcolorbox}}\fi

\renewcommand*\contentsname{Índice}
{
\hypersetup{linkcolor=}
\setcounter{tocdepth}{3}
\tableofcontents
}
code-block-bg: true

Organização e Apoio

\includegraphics{img/organizacao.png}

Patrocínio

\includegraphics{img/patrocinio.png}

\hypertarget{introduuxe7uxe3o}{%
\subsection{Introdução}\label{introduuxe7uxe3o}}

\hypertarget{conceitos-buxe1sicos}{%
\subsubsection{Conceitos Básicos}\label{conceitos-buxe1sicos}}

\hypertarget{requisitos}{%
\subsubsection{Requisitos}\label{requisitos}}

R e RStudio Instalado (link para instalação)

Python Instalado (inserir tutorial de instalação através do r)

ChatGTP API: para usar a api do chatgpt, você deve
\href{https://platform.openai.com/}{se cadastrar}. Além disso você deve
adicionar créditos para conseguir utilizar a API da OpenAI. Entre
\href{https://platform.openai.com/settings/organization/billing/overview}{nesta
página} e adicione. Sugiro adicionar US\$ 1, que será mais do que
suficiente para execução deste tutorial.

\hypertarget{conexuxe3o-com-a-api-da-openai}{%
\subsection{Conexão com a API da
OpenAI}\label{conexuxe3o-com-a-api-da-openai}}

Na \href{https://platform.openai.com/api-keys}{área de API Keys} você
deve criar nova chave usando o botão \emph{``Create new secret key''}.
Salve essa chave, que será usada neste tutorial.

Use o código abaixo para salvar a chave como uma variável de ambiente do
RStudio.

\begin{codelisting}

\caption{\texttt{R}}

\begin{Shaded}
\begin{Highlighting}[]
\FunctionTok{Sys.setenv}\NormalTok{(}\StringTok{\textasciigrave{}}\AttributeTok{OPENAI\_API\_KEY}\StringTok{\textasciigrave{}}\OtherTok{=} \StringTok{"COLE SUA CHAVE AQUI"}\NormalTok{)}
\end{Highlighting}
\end{Shaded}

\end{codelisting}

\hypertarget{configurauxe7uxe3o-de-ambiente-para-rodar-python-no-rstudio}{%
\subsection{Configuração de ambiente para rodar Python no
RStudio}\label{configurauxe7uxe3o-de-ambiente-para-rodar-python-no-rstudio}}

\begin{codelisting}

\caption{\texttt{R}}

\begin{Shaded}
\begin{Highlighting}[]
\FunctionTok{library}\NormalTok{(reticulate)}

\FunctionTok{virtualenv\_create}\NormalTok{(}\AttributeTok{envname =} \StringTok{"langchain\_rag\_pdf"}\NormalTok{,}
                  \AttributeTok{packages =} \FunctionTok{c}\NormalTok{( }\StringTok{"langchain"}\NormalTok{, }\StringTok{"openai"}\NormalTok{, }\StringTok{"pypdf"}\NormalTok{, }\StringTok{"bs4"}\NormalTok{,}
                                \StringTok{"python{-}dotenv"}\NormalTok{, }\StringTok{"chromadb"}\NormalTok{, }\StringTok{"tiktoken"}\NormalTok{,}
                                \StringTok{"langchain{-}openai"}\NormalTok{, }\StringTok{"langchain{-}community"}\NormalTok{))}

\NormalTok{reticulate}\SpecialCharTok{::}\FunctionTok{use\_virtualenv}\NormalTok{(}\StringTok{"langchain\_rag\_pdf"}\NormalTok{)}

\NormalTok{reticulate}\SpecialCharTok{::}\FunctionTok{py\_run\_string}\NormalTok{(}\StringTok{\textquotesingle{}}
\StringTok{print("Hello, world!") }
\StringTok{\textquotesingle{}}\NormalTok{)}

\NormalTok{api\_key\_for\_py }\OtherTok{\textless{}{-}} \FunctionTok{r\_to\_py}\NormalTok{(}\FunctionTok{Sys.getenv}\NormalTok{(}\StringTok{"OPENAI\_API\_KEY"}\NormalTok{))}
\end{Highlighting}
\end{Shaded}

\end{codelisting}

\hypertarget{download-e-importauxe7uxe3o-do-arquivo-em-pdf}{%
\subsection{Download e importação do arquivo em
PDF}\label{download-e-importauxe7uxe3o-do-arquivo-em-pdf}}

\begin{codelisting}

\caption{\texttt{R}}

\begin{Shaded}
\begin{Highlighting}[]
\ControlFlowTok{if}\NormalTok{(}\SpecialCharTok{!}\NormalTok{(}\FunctionTok{dir.exists}\NormalTok{(}\StringTok{"docs"}\NormalTok{))) \{}
  \FunctionTok{dir.create}\NormalTok{(}\StringTok{"docs"}\NormalTok{)}
\NormalTok{\}}

\FunctionTok{download.file}\NormalTok{(}\StringTok{"https://cran.r{-}project.org/web/packages/ggplot2/ggplot2.pdf"}\NormalTok{,}
              \AttributeTok{destfile =} \StringTok{"docs/ggplot2.pdf"}\NormalTok{, }\AttributeTok{mode =} \StringTok{"wb"}\NormalTok{)}
\end{Highlighting}
\end{Shaded}

\end{codelisting}

\begin{codelisting}

\caption{\texttt{Python}}

\begin{Shaded}
\begin{Highlighting}[]
\ImportTok{from}\NormalTok{ langchain.document\_loaders }\ImportTok{import}\NormalTok{ PyPDFLoader}

\NormalTok{my\_loader }\OperatorTok{=}\NormalTok{ PyPDFLoader(}\StringTok{\textquotesingle{}docs/ggplot2.pdf\textquotesingle{}}\NormalTok{)}
\BuiltInTok{print}\NormalTok{(}\BuiltInTok{type}\NormalTok{(my\_loader))}

\NormalTok{all\_pages }\OperatorTok{=}\NormalTok{ my\_loader.load()}

\BuiltInTok{print}\NormalTok{(}\BuiltInTok{type}\NormalTok{(all\_pages)) }

\BuiltInTok{print}\NormalTok{(}\BuiltInTok{len}\NormalTok{(all\_pages))}
\end{Highlighting}
\end{Shaded}

\end{codelisting}

\begin{codelisting}

\caption{\texttt{R}}

\begin{Shaded}
\begin{Highlighting}[]
\CommentTok{\# carregar documento em R}

\NormalTok{all\_pages\_in\_r }\OtherTok{\textless{}{-}}\NormalTok{ py}\SpecialCharTok{$}\NormalTok{all\_pages}

\NormalTok{all\_pages\_in\_r[[}\DecValTok{1}\NormalTok{]]}\SpecialCharTok{$}\NormalTok{metadata }\CommentTok{\# metadatos do primeiro item}
\FunctionTok{nchar}\NormalTok{(all\_pages\_in\_r[[}\DecValTok{100}\NormalTok{]]}\SpecialCharTok{$}\NormalTok{page\_content) }\CommentTok{\# conta a quantidade de caracteres do centésimo item}
\end{Highlighting}
\end{Shaded}

\end{codelisting}

\hypertarget{divisuxe3o-do-documento-em-pedauxe7os}{%
\subsection{Divisão do documento em
pedaços}\label{divisuxe3o-do-documento-em-pedauxe7os}}

\begin{codelisting}

\caption{\texttt{Python}}

\begin{Shaded}
\begin{Highlighting}[]
\ImportTok{import}\NormalTok{ openai}
\NormalTok{openai.api\_key }\OperatorTok{=}\NormalTok{ r.api\_key\_for\_py  }
\ImportTok{from}\NormalTok{ langchain.text\_splitter }\ImportTok{import}\NormalTok{ RecursiveCharacterTextSplitter}
\NormalTok{my\_doc\_splitter\_recursive }\OperatorTok{=}\NormalTok{ RecursiveCharacterTextSplitter()}
\NormalTok{my\_split\_docs }\OperatorTok{=}\NormalTok{ my\_doc\_splitter\_recursive.split\_documents(all\_pages)}
\end{Highlighting}
\end{Shaded}

\end{codelisting}

\begin{codelisting}

\caption{\texttt{R}}

\begin{Shaded}
\begin{Highlighting}[]
\NormalTok{my\_split\_docs }\OtherTok{\textless{}{-}}\NormalTok{ py}\SpecialCharTok{$}\NormalTok{my\_split\_docs}
\end{Highlighting}
\end{Shaded}

\end{codelisting}

\hypertarget{avaliauxe7uxe3o-do-custo-da-aplicauxe7uxe3o}{%
\subsection{Avaliação do custo da
aplicação}\label{avaliauxe7uxe3o-do-custo-da-aplicauxe7uxe3o}}

\begin{codelisting}

\caption{\texttt{R}}

\begin{Shaded}
\begin{Highlighting}[]
\NormalTok{my\_split\_docs }\OtherTok{\textless{}{-}}\NormalTok{ py}\SpecialCharTok{$}\NormalTok{my\_split\_docs}
\FunctionTok{length}\NormalTok{(my\_split\_docs)}

\NormalTok{total\_tokens }\OtherTok{\textless{}{-}}\NormalTok{ purrr}\SpecialCharTok{::}\FunctionTok{map\_int}\NormalTok{(my\_split\_docs, }
                               \SpecialCharTok{\textasciitilde{}}\NormalTok{ TheOpenAIR}\SpecialCharTok{::}\FunctionTok{count\_tokens}\NormalTok{(.x}\SpecialCharTok{$}\NormalTok{page\_content)) }\SpecialCharTok{|\textgreater{}} 
  \FunctionTok{sum}\NormalTok{()}
\end{Highlighting}
\end{Shaded}

\end{codelisting}

Atualmente (agosto de 2024), o custo do modelo \textbf{ada v2} usado
para criação dos \emph{embeddings} da OpenAI é de US\$ 0,10 / 1M tokens.
Como temos 153.172 tokens, o custo dessa etapa será de aproximadamente
US\$ 0,0153172.

\hypertarget{gerauxe7uxe3o-de-embeddings}{%
\subsection{\texorpdfstring{Geração de
\emph{embeddings}}{Geração de embeddings}}\label{gerauxe7uxe3o-de-embeddings}}

\begin{codelisting}

\caption{\texttt{R}}

\begin{Shaded}
\begin{Highlighting}[]
\ControlFlowTok{if}\NormalTok{(}\SpecialCharTok{!}\FunctionTok{dir.exists}\NormalTok{(}\StringTok{"docs/chroma\_db"}\NormalTok{)) \{}
  \FunctionTok{dir.create}\NormalTok{(}\StringTok{"docs/chroma\_db"}\NormalTok{)}
\NormalTok{\}}
\end{Highlighting}
\end{Shaded}

\end{codelisting}

\begin{codelisting}

\caption{\texttt{Python}}

\begin{Shaded}
\begin{Highlighting}[]
\ImportTok{import}\NormalTok{ os}
\NormalTok{os.environ[}\StringTok{"OPENAI\_API\_KEY"}\NormalTok{] }\OperatorTok{=}\NormalTok{ r.api\_key\_for\_py }

\ImportTok{from}\NormalTok{ langchain\_openai }\ImportTok{import}\NormalTok{ OpenAIEmbeddings}
\NormalTok{embed\_object }\OperatorTok{=}\NormalTok{ OpenAIEmbeddings()}

\ImportTok{from}\NormalTok{ langchain\_community.vectorstores }\ImportTok{import}\NormalTok{ Chroma}
\NormalTok{chroma\_store\_directory }\OperatorTok{=} \StringTok{"docs/chroma\_db"}

\NormalTok{vectordb }\OperatorTok{=}\NormalTok{ Chroma.from\_documents(}
\NormalTok{    documents}\OperatorTok{=}\NormalTok{my\_split\_docs,}
\NormalTok{    embedding}\OperatorTok{=}\NormalTok{embed\_object,}
\NormalTok{    persist\_directory}\OperatorTok{=}\NormalTok{chroma\_store\_directory}
\NormalTok{)}

\CommentTok{\# numero de embeddings criados}
\BuiltInTok{print}\NormalTok{(vectordb.\_collection.count()) }
\end{Highlighting}
\end{Shaded}

\end{codelisting}

\hypertarget{gerauxe7uxe3o-de-respostas-uxfanicas}{%
\subsection{Geração de respostas
únicas}\label{gerauxe7uxe3o-de-respostas-uxfanicas}}

\hypertarget{carregar-embeddings}{%
\subsubsection{Carregar embeddings}\label{carregar-embeddings}}

\begin{codelisting}

\caption{\texttt{Python}}

\begin{Shaded}
\begin{Highlighting}[]
\CommentTok{\# carregar embeddings}

\ImportTok{import}\NormalTok{ openai}
\NormalTok{openai.api\_key }\OperatorTok{=}\NormalTok{ r.api\_key\_for\_py  }
\ImportTok{from}\NormalTok{ langchain\_community.embeddings }\ImportTok{import}\NormalTok{ OpenAIEmbeddings}
\NormalTok{embed\_object }\OperatorTok{=}\NormalTok{ OpenAIEmbeddings()}

\ImportTok{from}\NormalTok{ langchain\_community.vectorstores }\ImportTok{import}\NormalTok{ Chroma}
\NormalTok{chroma\_store\_directory }\OperatorTok{=} \StringTok{"docs/chroma\_db"}
\NormalTok{vectordb }\OperatorTok{=}\NormalTok{ Chroma(persist\_directory}\OperatorTok{=}\NormalTok{chroma\_store\_directory, }
\NormalTok{                  embedding\_function}\OperatorTok{=}\NormalTok{embed\_object)}
\end{Highlighting}
\end{Shaded}

\end{codelisting}

\hypertarget{definir-qual-modelo-de-linguagem-seruxe1-usado}{%
\subsubsection{Definir qual modelo de linguagem será
usado}\label{definir-qual-modelo-de-linguagem-seruxe1-usado}}

\begin{codelisting}

\caption{\texttt{Python}}

\begin{Shaded}
\begin{Highlighting}[]
\CommentTok{\# Set up the LLM you want to use, in this example OpenAI\textquotesingle{}s gpt{-}3.5{-}turbo}
\CommentTok{\# from langchain.chat\_models import ChatOpenAI}
\ImportTok{from}\NormalTok{ langchain\_community.chat\_models }\ImportTok{import}\NormalTok{ ChatOpenAI}
\ImportTok{from}\NormalTok{ langchain\_openai }\ImportTok{import}\NormalTok{ ChatOpenAI}
\NormalTok{the\_llm }\OperatorTok{=}\NormalTok{ ChatOpenAI(model\_name}\OperatorTok{=}\StringTok{"gpt{-}3.5{-}turbo"}\NormalTok{, temperature}\OperatorTok{=}\DecValTok{0}\NormalTok{)}

\CommentTok{\# Create a chain using the RetrievalQA component}
\ImportTok{from}\NormalTok{ langchain.chains }\ImportTok{import}\NormalTok{ RetrievalQA}
\NormalTok{qa\_chain }\OperatorTok{=}\NormalTok{ RetrievalQA.from\_chain\_type(the\_llm, retriever}\OperatorTok{=}\NormalTok{vectordb.as\_retriever())}
\end{Highlighting}
\end{Shaded}

\end{codelisting}

\hypertarget{obtenuxe7uxe3o-da-resposta}{%
\subsubsection{Obtenção da resposta}\label{obtenuxe7uxe3o-da-resposta}}

\begin{codelisting}

\caption{\texttt{Python}}

\begin{Shaded}
\begin{Highlighting}[]
\NormalTok{my\_question }\OperatorTok{=} \StringTok{"How do you rotate text on the x{-}axis of a graph?"}
\BuiltInTok{print}\NormalTok{(qa\_chain.invoke(my\_question))}
\end{Highlighting}
\end{Shaded}

\end{codelisting}

\hypertarget{gerauxe7uxe3o-de-respostas-como-em-uma-conversa}{%
\subsection{Geração de respostas como em uma
conversa}\label{gerauxe7uxe3o-de-respostas-como-em-uma-conversa}}

\hypertarget{junuxe7uxe3o-da-consulta-com-os-pedauxe7os-do-documento}{%
\subsubsection{Junção da consulta com os pedaços do
documento}\label{junuxe7uxe3o-da-consulta-com-os-pedauxe7os-do-documento}}

\begin{codelisting}

\caption{\texttt{Python}}

\begin{Shaded}
\begin{Highlighting}[]
\ImportTok{from}\NormalTok{ langchain\_core.prompts }\ImportTok{import}\NormalTok{ ChatPromptTemplate, MessagesPlaceholder}
\ImportTok{from}\NormalTok{ langchain.chains }\ImportTok{import}\NormalTok{ create\_retrieval\_chain}
\ImportTok{from}\NormalTok{ langchain.chains.combine\_documents }\ImportTok{import}\NormalTok{ create\_stuff\_documents\_chain}
\ImportTok{from}\NormalTok{ langchain\_core.runnables.history }\ImportTok{import}\NormalTok{ RunnableWithMessageHistory}
\ImportTok{from}\NormalTok{ langchain\_core.chat\_history }\ImportTok{import}\NormalTok{ BaseChatMessageHistory}
\ImportTok{from}\NormalTok{ langchain\_community.chat\_message\_histories }\ImportTok{import}\NormalTok{ ChatMessageHistory}
\ImportTok{from}\NormalTok{ langchain\_core.prompts }\ImportTok{import}\NormalTok{ ChatPromptTemplate, MessagesPlaceholder}

\NormalTok{retriever }\OperatorTok{=}\NormalTok{ vectordb.as\_retriever()}

\NormalTok{llm }\OperatorTok{=}\NormalTok{ ChatOpenAI(model}\OperatorTok{=}\StringTok{"gpt{-}3.5{-}turbo"}\NormalTok{, temperature}\OperatorTok{=}\DecValTok{0}\NormalTok{)}

\CommentTok{\#\#\# Contextualize question }\AlertTok{\#\#\#}
\NormalTok{contextualize\_q\_system\_prompt }\OperatorTok{=}\NormalTok{ (}
    \StringTok{"Given a chat history and the latest user question "}
    \StringTok{"which might reference context in the chat history, "}
    \StringTok{"formulate a standalone question which can be understood "}
    \StringTok{"without the chat history. Do NOT answer the question, "}
    \StringTok{"just reformulate it if needed and otherwise return it as is."}
\NormalTok{)}

\NormalTok{contextualize\_q\_prompt }\OperatorTok{=}\NormalTok{ ChatPromptTemplate.from\_messages(}
\NormalTok{    [}
\NormalTok{        (}\StringTok{"system"}\NormalTok{, contextualize\_q\_system\_prompt),}
\NormalTok{        MessagesPlaceholder(}\StringTok{"chat\_history"}\NormalTok{),}
\NormalTok{        (}\StringTok{"human"}\NormalTok{, }\StringTok{"}\SpecialCharTok{\{input\}}\StringTok{"}\NormalTok{),}
\NormalTok{    ]}
\NormalTok{)}

\NormalTok{history\_aware\_retriever }\OperatorTok{=}\NormalTok{ create\_history\_aware\_retriever(}
\NormalTok{    llm, retriever, contextualize\_q\_prompt}
\NormalTok{)}
\end{Highlighting}
\end{Shaded}

\end{codelisting}

\hypertarget{construuxe7uxe3o-da-resposta}{%
\subsubsection{Construção da
resposta}\label{construuxe7uxe3o-da-resposta}}

\begin{codelisting}

\caption{\texttt{Pyhton}}

\begin{Shaded}
\begin{Highlighting}[]
\NormalTok{system\_prompt }\OperatorTok{=}\NormalTok{ (}
    \StringTok{"You are an assistant for question{-}answering tasks. "}
    \StringTok{"Use the following pieces of retrieved context to answer "}
    \StringTok{"the question. If you don\textquotesingle{}t know the answer, say that you "}
    \StringTok{"don\textquotesingle{}t know. If the question is out of context of the "}
    \StringTok{"retrieved context, do not answerr and just say it is out of context.}\CharTok{\textbackslash{}n\textbackslash{}n}\StringTok{"}
    \StringTok{"}\SpecialCharTok{\{context\}}\StringTok{"}
\NormalTok{)}

\NormalTok{qa\_prompt }\OperatorTok{=}\NormalTok{ ChatPromptTemplate.from\_messages(}
\NormalTok{    [}
\NormalTok{        (}\StringTok{"system"}\NormalTok{, system\_prompt),}
\NormalTok{        MessagesPlaceholder(}\StringTok{"chat\_history"}\NormalTok{),}
\NormalTok{        (}\StringTok{"human"}\NormalTok{, }\StringTok{"}\SpecialCharTok{\{input\}}\StringTok{"}\NormalTok{),}
\NormalTok{    ]}
\NormalTok{)}

\NormalTok{question\_answer\_chain }\OperatorTok{=}\NormalTok{ create\_stuff\_documents\_chain(llm, qa\_prompt)}

\NormalTok{rag\_chain }\OperatorTok{=}\NormalTok{ create\_retrieval\_chain(history\_aware\_retriever, question\_answer\_chain)}
\end{Highlighting}
\end{Shaded}

\end{codelisting}

\hypertarget{gerauxe7uxe3o-da-resposta-atravuxe9s-de-uma-conversa}{%
\subsubsection{Geração da resposta através de uma
conversa}\label{gerauxe7uxe3o-da-resposta-atravuxe9s-de-uma-conversa}}

\begin{codelisting}

\caption{\texttt{Python}}

\begin{Shaded}
\begin{Highlighting}[]
\NormalTok{store }\OperatorTok{=}\NormalTok{ \{\}}

\KeywordTok{def}\NormalTok{ get\_session\_history(session\_id: }\BuiltInTok{str}\NormalTok{) }\OperatorTok{{-}\textgreater{}}\NormalTok{ BaseChatMessageHistory:}
    \ControlFlowTok{if}\NormalTok{ session\_id }\KeywordTok{not} \KeywordTok{in}\NormalTok{ store:}
\NormalTok{        store[session\_id] }\OperatorTok{=}\NormalTok{ ChatMessageHistory()}
    \ControlFlowTok{return}\NormalTok{ store[session\_id]}

\NormalTok{conversational\_rag\_chain }\OperatorTok{=}\NormalTok{ RunnableWithMessageHistory(}
\NormalTok{    rag\_chain,}
\NormalTok{    get\_session\_history,}
\NormalTok{    input\_messages\_key}\OperatorTok{=}\StringTok{"input"}\NormalTok{,}
\NormalTok{    history\_messages\_key}\OperatorTok{=}\StringTok{"chat\_history"}\NormalTok{,}
\NormalTok{    output\_messages\_key}\OperatorTok{=}\StringTok{"answer"}\NormalTok{,}
\NormalTok{)}

\KeywordTok{def}\NormalTok{ get\_answer(question, session\_id }\OperatorTok{=} \StringTok{"abc123"}\NormalTok{):}
\NormalTok{  result }\OperatorTok{=}\NormalTok{ conversational\_rag\_chain.invoke(}
\NormalTok{    \{}\StringTok{"input"}\NormalTok{: question\},}
\NormalTok{    config}\OperatorTok{=}\NormalTok{\{}\StringTok{"configurable"}\NormalTok{: \{}\StringTok{"session\_id"}\NormalTok{: session\_id\}\},}
\NormalTok{  )}
  
  \ControlFlowTok{return}\NormalTok{ result}
\end{Highlighting}
\end{Shaded}

\end{codelisting}

\hypertarget{usando-o-chat-via-python}{%
\subsubsection{Usando o chat via
Python}\label{usando-o-chat-via-python}}

\begin{codelisting}

\caption{\texttt{Python}}

\begin{Shaded}
\begin{Highlighting}[]
\ControlFlowTok{while} \VariableTok{True}\NormalTok{:}
\NormalTok{  my\_question }\OperatorTok{=} \BuiltInTok{input}\NormalTok{(}\StringTok{"Diga: "}\NormalTok{)}
  
  \ControlFlowTok{if}\NormalTok{ my\_question.lower() }\OperatorTok{==} \StringTok{\textquotesingle{}sair\textquotesingle{}}\NormalTok{:}
    \BuiltInTok{print}\NormalTok{(}\StringTok{"Encerrando o chatbot."}\NormalTok{)}
    \ControlFlowTok{break}
          
\NormalTok{  result }\OperatorTok{=}\NormalTok{ get\_answer(my\_question, }\StringTok{"abc123"}\NormalTok{)}
  
  \BuiltInTok{print}\NormalTok{(result[}\StringTok{"answer"}\NormalTok{])}
\end{Highlighting}
\end{Shaded}

\end{codelisting}

\hypertarget{usando-o-chat-via-r}{%
\subsubsection{Usando o chat via R}\label{usando-o-chat-via-r}}

\begin{Shaded}
\begin{Highlighting}[]
\FunctionTok{py\_run\_string}\NormalTok{(}\StringTok{\textquotesingle{}}
\StringTok{print(qa\_chain.run("How can I make a bar chart where the bars are steel blue? Answer in pt{-}br"))}
\StringTok{\textquotesingle{}}\NormalTok{)}

\FunctionTok{py\_run\_string}\NormalTok{(}\StringTok{\textquotesingle{}}
\StringTok{print(qa\_chain.run("What is the capital of Australia?"))}
\StringTok{\textquotesingle{}}\NormalTok{)}
\end{Highlighting}
\end{Shaded}




\end{document}
